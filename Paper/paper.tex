
\documentclass{article}
\usepackage{ijcai11}
\usepackage{times}
\usepackage{gensymb}


\title{Smart LED-displays}
\author{Barbara Ameloot\\
KU Leuven\\
barbara.ameloot@student.kuleuven.be
\And 
Wouter Jochems\\
KU Leuven\\
wouter.jochems@student.kuleuven.be}


\begin{document}

\maketitle

\begin{abstract}
  Deze paper onderzoekt een mogelijke toepassing van smart LEDs, LEDs aangestuurd door een microcontroller. Specifiek bestuderen we hun mogelijkheden op het gebied van LED-displays. LED-displays komen voor in alle soorten en maten. Helaas zijn deze niet altijd perfect. Het gebruik van smart LEDs zou hier verandering in kunnen brengen. We onderzoeken de mogelijkheid dat de afzonderlijke LEDs van het display met elkaar kunnen communiceren alsook een opstelling waarbij de LEDs signalen krijgen van een externe bron. We concluderen dat vlakke displays voorkomen dat de LEDs zijdelings kunnen communiceren.
\end{abstract}


\section{Introductie}

LED-displays worden tegenwoordig vaak gebruikt op evenementen en reclameborden. Deze werken steeds met een centrale verwerkingseenheid. Als er problemen opduiken met deze eenheid of met de verbinding tussen de LEDs en de controller, dan wordt meteen een groot deel van het scherm buiten werking. Ook beperkt de nood aan verbinding tussen controller en LEDs de mogelijkheden van het display.
Smart LEDs kunnen een oplossing bieden voor deze problemen aangezien elke LED over zijn eigen microcontroller beschikt. Dit voorkomt dat bij een fout een groot deel van het display onbruikbaar wordt. Ook kunnen de verschillende LEDs ten opzichte van elkaar bewegen zonder dat er kabels in de knoop raken of radiogolven met elkaar interfereren. Daarnaast kunnen de smart LEDs met elkaar communiceren en zijn de LEDs dus bewust van elkaar.
Smart LEDs communiceren met elkaar via VLC (Visual Light Communication). De ingebouwde LED dient als zender en als ontvanger.


\section{Probleemstelling}

We willen uiteindelijk een LED-display bekomen waarbij de LEDs onderling kunnen communiceren zonder id’s, reageren op een externe lichtbron en blijven werken onder alle lichtomstandigheden. Daarnaast stellen we ons de vraag of een smart LED-display een meerwaarde heeft ten op zichtte van een gewoon LED-display. Tijdens de uitvoering van ons project kwamen nog enkele andere interessante problemen boven die verder in deze tekst besproken worden.


\section{Implementatie}
Om onze hypothese te testen maakten we gebruik van LEDs verbonden met Zigduino r2 microcontrollers. We maakten gebruik van infrarood LEDs voor de communicatie om extra licht flikkeringen in het display te voorkomen en om interferentie van de omgeving zo veel mogelijk te vermijden.


\section{Oplossing}

\subsection{Onderlinge communicatie}
Om een vlak display te bekomen, moeten de LEDs uiteraard zijdelings staan. Dit zorgde helaas voor problemen, aangezien elke LED een specifieke invalshoek heeft. Wij hadden echter geen LEDs met een hoek van 180\degree ter beschikking. Hierdoor lukte het ons niet om zijdelings een signaal door te zenden. 
Met twee naar elkaar gerichte LEDs lukt het ons wel om een bit-string door te zenden. Onze opstelling bestaat uit een eerste LED die een bit-string uitzend, een tweede die deze ontvangt en doorgeeft aan een microcontroller en vervolgens een derde die de ontvangen bit-string weer uitzend in een andere richting. 
In minder conventionele LED-displays (met een vorm die het toelaat de LEDs naar een elkaar te richten of met bewegende onderdelen) is onderlinge communicatie dus wel mogelijk.

\subsection{Externe lichtbron}
Omdat een tweede LED ook een externe lichtbron is, valt uit het voorgaande al te besluiten dat een goed gericht licht in staat is om te interageren met een smart LED-display. Wij maakten gebruik van een afstandsbediening die naargelang welke knop werd ingeduwd, een ander stukje code uitzond.

\subsection{Omgevingslicht}
Om ervoor te zorgen dat het programma meer omgevingslicht niet als een signaal zou herkennen, voegden we een dynamische threshold toe.
Onze huidige aanpak neemt bij elke iteratie het gemiddelde van een tiental metingen. Deze waarde wordt vergeleken met de vorige threshold, zo wordt vermeden dat een te hoge waarde wordt aangenomen als er tijdens de metingen al een signaal verzonden wordt.
We merkten op dat bij deze aanpak de threshold zeer veel veranderde. Bij het visualiseren van deze ruis was een duidelijk sinusoïdaal verloop te zien. Dit is te verklaren aan de hand van de spoelen en condensatoren in de microcontroller die constant opladen en ontladen. Eventueel kan hier rekening mee gehouden worden om een nog accuratere threshold te bekomen.

\subsection{Meerdere LEDs}
Bij aanvang van ons project hoopten we voor het display en het verzenden van informatie dezelfde LED te gebruiken. Dit bracht echter meer moeilijkheden met zich mee, zoals informatie die verzonden moest worden over een LED die volgens het display geen licht mag uitzenden. Hierdoor kozen we om met een tweede LED te werken. Zo bestaat onze opstelling dus uit een infrarood LED om signalen te ontvangen en een RGB LED om licht uit te zenden voor het display.


\section{Evaluatie}


\section{Conclusie}


\section{Verwant werk}
Dit project is een vervolg van een project van vorig jaar waarin de smart LEDs ontwikkeld werden. 


\section{Verder werk}
Als vervolg op dit project zou een display opgebouwd uit echte smart LEDs gemaakt kunnen worden. Het zou ook interessant zijn mocht er een oplossing gevonden worden voor het probleem met de invalshoek van de LEDs zodat zijdelings gecommuniceerd kan worden. Verder is er nog de reeds voorgestelde aanpassing aan de threshold die rekening zou kunnen houden met de sinusoïdale ruis.


%\bibliographystyle{named}
%\bibliography{ijcai11}

\end{document}

